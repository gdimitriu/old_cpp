\chapter{Introducere}

\hspace{5mm}Acesta este implementarea programului rshell (Remote Shell)
pentru masinile UNIX. El ne permite sa executam comenzi pe masina target,
aceste sunt comenzi Bash, returnind apoi rezultatul pe masina host. Este un
program mai simplu decit telnet-ul.

Limitarile sunt:
\begin{itemize}
\item Nu suporta login, su si alte schimbari de user.
\item Nesuportind optiunile spuse mai inainte se poatevedea ca userul care
porneste rshell-ul are accesul ingradit sau are complet accesul in functie
de userul care a lansat serverul pe masina target.
\item Nu suprota apelul de programe care lucreaza grafic sau cu consola, de
exemplu nu se poate lansa un program X, el se lanseaza dar ruleaza pe masina
target si pierdem controlul lui.
\item Nu suporta aducerea de fisiere, deoarece a fost gindit ca un program
care ruleaza numai pe masina target.
\end{itemize}

In toata lucrarea se numeste {\bf masina target} masina care ruleaza
servarul si {\bf masina host} masina de pe care se lanseaza rshell-ul adica
clientul.

Programul este implementat in ANSI C, pentru masinile UNIX, avind o
portabilitate maxima. Cerinta este ca masina target sa aiba instalat
interpretorul {\bf bash} pe care se bazeaza programul si sa aiba un
compilator compatibil gcc. Ca metoda de comunicare cu reteaua s-a am
stabilit ca se va lucra pe socketuri cu conexiune adica TCP, deoarece este
mai sigur, programul neavind confirmari pe parcurs.

Comenzile se introduc dupa apelarea programului cu {\bf rshell "IP\_adress"}
si trebuie neaparat ca intre comanda si argument sa se lase un spatiu.
Pentru iesiere se tasteaza exit. Daca se observa ca listingul rezultat este
prea lung se poate face un pipe catre orice pager dar se incurajeaza pagerul
less.

Servarul din constructie nu accepta decit un singur client.